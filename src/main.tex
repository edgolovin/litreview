\documentclass[11pt]{article}
\usepackage{geometry}
 \geometry{
 a4paper,
 left=30mm,
 right=20mm,
 top=20mm,
 bottom=20mm,
 }
\usepackage[T2A]{fontenc}            % внутренняя кодировка  TeX
\usepackage[utf8x]{inputenc}         % кодовая страница документа
\usepackage[english, russian]{babel} % локализация и переносы
\usepackage{indentfirst}
\usepackage{graphicx}
%\graphicspath{ {./src/img/} }

\setlength{\parindent}{4em}

%\renewcommand{\rmdefault}{cmr}
\renewcommand{\baselinestretch}{1.1}


\title{Образ Китая в СМИ: методологический обзор диссертационных исследований на русском языке}
\author{Янь Цинвэнь}
%\date{10 октября 2020 г.}


\begin{document}
\maketitle

\begin{abstract}

Статья посвящена анализу диссертационных исследований на русском языке, посвященных образу Китая в СМИ.

\end{abstract}

\large
\section{Введение}\label{sec:intro}

В данной статье рассматриваются диссертационные работы, посвященные теме ``Образ Китая в СМИ''.
Поисковая выдача в электронном каталоге РГБ по ключевой фразе ``Образ Китая'' содержит 22 диссертационных исследования с 1991 по 2020 год.
Из найденных работ мы отобрали только 6 исследований, в которых материалом для изучения являются тексты современных СМИ.
Проанализируем методологические подходы авторов данных исследований к изучению материала.

\small
\begin{center}
    \begin{tabular}{ p{0.07\linewidth} p{0.2\linewidth} p{0.65\linewidth} }
        2007 & Сорокина О.Н. & Языковая реализация образа Китая как информационной модели в средствах массовой информации США \\
        2011 & Монастырева О.В. & Методы продвижения образа страны в практике «Международного радио Китая»: функциональные и творческие характеристики вещания на русском языке \\
        2012 & Ван Сюй & Китай в печатных российских СМИ (номинативный аспект) \\
        2017 & Ду Цюаньбо & Особенности имиджа КНР в русскоязычных печатно-сетевых СМИ Китая \\
        2017 & Шао Дэвань & Отражение российско-китайских отношений в русскоязычных СМИ КНР 2006--2016 гг. (образ Китая) \\
        2020 & Чэн Юйсяо & Образ Китая в российских СМИ: лингвокогнитивный аспект \\
    \end{tabular}
\end{center}	

\large

Из шести рассматриваемых работ три в качестве материала для исследований используют публикации на русском языке китайских СМИ.
Еще две работы - публикации российских СМИ.
И одна работа - публикации американских СМИ на английском языке.
Казалось бы, все исследователи обращают свой взгляд на один и тот же предмет: образ Китая в СМИ.
Однако, каждый исследователь рассматривает образ с позиций разных реципиентов и интересантов, разных СМИ, разных аспектов самого образа.
Таким образом, авторы используют разные методологические подходы к его изучению.

Охарактеризуем методологические особенности каждого исследования, основываясь на текстах диссертаций или их авторефератов.

\section{Диссертация Сорокиной О.Н.}\label{sec:section-sorokina}

Образ Китая создаваемый кем, для кого, с какой целью?
- Американским правящим классом, для американского народа, с целью манипулятивного влияния на картину мира американцев.

Образ Китая формируется посредством чего?
- Посредством языковых средств в публикациях массовых американских СМИ.

Исследуются 260 статей в американских СМИ за пятилетний период (2001--2006).
На основе выбранного материала, делаются выводы о создании образа Китая в глазах американцев, представляющего угрозу, имеющего большие экономические и военные аппетиты, угрожающие благополучию не только США, но и всего мира.
Автор выделяет несколько механизмов построения подобного образа через построение манипулятивной картины мира на основе таких фрагментов образа Китая, как: 1) экономика Китая;
2) идеология Китая;
3) нарушение прав человека в Китае;
4) вооружение Китая.
Все указанные компоненты используются исключительно как негативные.

Ведущими языковыми приемами для реализации негативного образа в американских СМИ, являются метафора и эпитет.

\section{Диссертация Монастыревой О.В.}\label{sec:section-monastyreva}

Образ Китая создаваемый кем, для кого, с какой целью?
- Китайским правительством, для российского народа, с целью создания условий для позитивной коммуникации Китая и России.

Образ Китая формируется посредством чего?
- Посредством языковых средств в передачах ``Международного радио Китая''.

Исследуются 684 текста передач ``Международного радио Китая'' за временной промежуток 2006--2010 гг.
Автор уделяет внимание "коммуникативным стратегиям" формирования и продвижения образа страны.
Основной предметной сферой русскоязычного вещания МРК является культура.
Все многообразие средств и методов убеждающей коммуникации направлено на создание положительных и блокировке негативных представлений о Китае.

Автор выявил наиболее частотные слова и устойчивые словосочетания, которые могут претендовать на роль лингвокультурных концептов.
Среди них: быт, труд, топос, этнокультурная идентичность, семья, искусство и просвещение, нормы и ценности.
Выявленные концепты полностью относятся автором к сфере культуры.

Еще добавить??

\section{Диссертация Ван Сюй}\label{sec:section-wangxue}

Образ Китая создаваемый кем, для кого, с какой целью?
- Семь российских изданий (проправительственные?), для российского народа, определена ли цель?

Образ Китая формируется посредством чего?
- Посредством единиц номинации, жанрового состава публикаций.

Исследуются более 2000 текстов, вышедших в печать с мая 2006 по ноябрь 2010 года в 7-ми российских изданиях.
На данном материале исследуются единицы номинации, представляющие Китай.
Исследование посвящено выявлению главных признаковых характеристик формируемого СМИ образа Китая.

Делается упор именно на печатные издания, что в 2012 году, и даже 2009 (начало аспирантуры автора диссертации), отделить чисто печатные издания от интернет-изданий часто уже невозможно.

Автор исследует "языковую картину мира", частью которой является и Китай, помещаемый в эту языковую картину через СМИ.
В российских СМИ одной из базовых схем номинаций является экономика Китая.
Российские СМИ используют в основном номинации с нейтральной экспрессивно-стилистической окраской.
Автор анализирует жанровый состав набранного материала.

\section{Диссертация Ду Цюаньбо}\label{sec:section-duquanbo}

Образ Китая создаваемый кем, для кого, с какой целью?
- Китайское правительство, для российского народа, определена ли цель?

Образ Китая формируется посредством чего?
- Посредством языковых средств (чаще всего фразеологизмов), жанрового состава публикаций, заголовков как концетраторов темы, концепции "мягкой силы" - культурного влияния.

Исследуются более 10 тысяч публикаций трех изданий (Азиатское иллюстрированное обозрение ``Россия---Китай'', журнал ``Китай'', газета ``Женьминь Жибао'' на русском языке) за период 2009--2016 гг.
Автор рассматривает китайские СМИ, издающиеся на русском языке, как один из компонентов системы СМИ КНР.
Предметом исследования являются жанровое, стилистическое и языковое своеобразие СМИ Китая, которые относятся к имиджеобразующим особенностям образа Китая в вышеобозначенных трех изданиях.
Автор обращает внимание на такие аспекты формирования имиджа, как наличие стереотипов, мифологизации.
Упоминается взятая на вооружение концепция "мягкой силы".

Автор считает наиболее существенной особенностью языка китайской журналистики, в т.ч. на русском языке, его книжность и высокопарность вследствие обилия т.н. вэньянизмов, клишированных стандартных конструкций.
Эта особенность лишает публикацию непосредственности и близости к аудитории.
Вэньянизмы - это языковые единицы различных уровней: лексики, синтаксиса, стилистики;
но чаще всего - это устойчивые выражения и фразеологизмы, унаследованные журналистикой из старого литературного языка.

Важное значение автор придает и жанровой палитре публикаций о Китае, рассматривает заголовок как концентратор темы.

Каждое издание рассматривается автором отдельно, но в итоге выводы сведены к тому, что специфика языка и стиля рассматриваемых СМИ обусловлены традициями китайской журналистики:
приверженность строгому литературному книжному стилю, в котором сочетаются стандартные выражения и экспрессивные средства образной выразительности.
Первые дают публикациям высокопарность, вторые - воздействующую на читателя эмоциональную составляющую.
Жанрово-тематическая заданность полностью соответствует имиджевой информационной политике Китая.

\section{Диссертация Шао Дэвань}\label{sec:section-shaodewan}

Образ Китая создаваемый кем, для кого, с какой целью?
- Китайское правительство, для российского народа, определена ли цель?

Образ Китая формируется посредством чего?
- ?????

Исследуются более 5 тысяч публикаций четырех изданий (ИА ``Синьхуа'' на русском языке, Азиатское иллюстрированное обозрение ``Россия--Китай'', журнал ``Китай'', газета ``Женьминь Жибао'' на русском языке)за период 2006--2016 гг.

Проверить плагиат на Ван Сюй.
В списке литературы нет плагиата.

Говняный дисер


\section{Диссертация Чэн Юйсяо}\label{sec:section-chengyuxiao}

Образ Китая создаваемый кем, для кого, с какой целью?
- Неизвестно.

Образ Китая формируется посредством чего?
- Посредством метафорических единиц.

Исследуются свыше 4 тысяч метафорических единиц из источников российских СМИ за период 2000--2019 гг, представленных в корпусах НКРЯ и Интегрум.

Говняный высер

    \bibliography{main}
    \bibliographystyle{plain}

\end{document}
